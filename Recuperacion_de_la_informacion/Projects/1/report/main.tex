%%%%%%%%%%%%%%%%%%%%%%%%%%%%%%%%%%%%%%%%%
% Lachaise Assignment
% LaTeX Template
% Version 1.0 (26/6/2018)
%
% This template originates from:
% http://www.LaTeXTemplates.com
%
% Authors:
% Marion Lachaise & François Févotte
% Vel (vel@LaTeXTemplates.com)
%
% License:
% CC BY-NC-SA 3.0 (http://creativecommons.org/licenses/by-nc-sa/3.0/)
%
%%%%%%%%%%%%%%%%%%%%%%%%%%%%%%%%%%%%%%%%%

%----------------------------------------------------------------------------------------
%	PACKAGES AND OTHER DOCUMENT CONFIGURATIONS
%----------------------------------------------------------------------------------------

\documentclass{article}

%%%%%%%%%%%%%%%%%%%%%%%%%%%%%%%%%%%%%%%%%
% Lachaise Assignment
% Structure Specification File
% Version 1.0 (26/6/2018)
%
% This template originates from:
% http://www.LaTeXTemplates.com
%
% Authors:
% Marion Lachaise & François Févotte
% Vel (vel@LaTeXTemplates.com)
%
% License:
% CC BY-NC-SA 3.0 (http://creativecommons.org/licenses/by-nc-sa/3.0/)
% 
%%%%%%%%%%%%%%%%%%%%%%%%%%%%%%%%%%%%%%%%%

%----------------------------------------------------------------------------------------
%	PACKAGES AND OTHER DOCUMENT CONFIGURATIONS
%----------------------------------------------------------------------------------------



\usepackage{amsmath,amsfonts,stmaryrd,amssymb} % Math packages

\usepackage{enumerate} % Custom item numbers for enumerations

\usepackage[ruled]{algorithm2e} % Algorithms

\usepackage[framemethod=tikz]{mdframed} % Allows defining custom boxed/framed environments

\usepackage{listings} % File listings, with syntax highlighting
\lstset{
	basicstyle=\ttfamily, % Typeset listings in monospace font
}

%----------------------------------------------------------------------------------------
%	DOCUMENT MARGINS
%----------------------------------------------------------------------------------------

\usepackage{geometry} % Required for adjusting page dimensions and margins

\geometry{
	paper=a4paper, % Paper size, change to letterpaper for US letter size
	top=2.5cm, % Top margin
	bottom=3cm, % Bottom margin
	left=2.5cm, % Left margin
	right=2.5cm, % Right margin
	headheight=14pt, % Header height
	footskip=1.5cm, % Space from the bottom margin to the baseline of the footer
	headsep=1.2cm, % Space from the top margin to the baseline of the header
	%showframe, % Uncomment to show how the type block is set on the page
}

%----------------------------------------------------------------------------------------
%	FONTS
%----------------------------------------------------------------------------------------

\usepackage[utf8]{inputenc} % Required for inputting international characters
\usepackage[T1]{fontenc} % Output font encoding for international characters

% \usepackage{XCharter} % Use the XCharter fonts

%----------------------------------------------------------------------------------------
%	COMMAND LINE ENVIRONMENT
%----------------------------------------------------------------------------------------

% Usage:
% \begin{commandline}
%	\begin{verbatim}
%		$ ls
%		
%		Applications	Desktop	...
%	\end{verbatim}
% \end{commandline}

\mdfdefinestyle{commandline}{
	leftmargin=10pt,
	rightmargin=10pt,
	innerleftmargin=15pt,
	middlelinecolor=black!50!white,
	middlelinewidth=2pt,
	frametitlerule=false,
	backgroundcolor=black!5!white,
	frametitle={Command Line},
	frametitlefont={\normalfont\sffamily\color{white}\hspace{-1em}},
	frametitlebackgroundcolor=black!50!white,
	nobreak,
}

% Define a custom environment for command-line snapshots
\newenvironment{commandline}{
	\medskip
	\begin{mdframed}[style=commandline]
}{
	\end{mdframed}
	\medskip
}

%----------------------------------------------------------------------------------------
%	FILE CONTENTS ENVIRONMENT
%----------------------------------------------------------------------------------------

% Usage:
% \begin{file}[optional filename, defaults to "File"]
%	File contents, for example, with a listings environment
% \end{file}

\mdfdefinestyle{file}{
	innertopmargin=1.6\baselineskip,
	innerbottommargin=0.8\baselineskip,
	topline=false, bottomline=false,
	leftline=false, rightline=false,
	leftmargin=2cm,
	rightmargin=2cm,
	singleextra={%
		\draw[fill=black!10!white](P)++(0,-1.2em)rectangle(P-|O);
		\node[anchor=north west]
		at(P-|O){\ttfamily\mdfilename};
		%
		\def\l{3em}
		\draw(O-|P)++(-\l,0)--++(\l,\l)--(P)--(P-|O)--(O)--cycle;
		\draw(O-|P)++(-\l,0)--++(0,\l)--++(\l,0);
	},
	nobreak,
}

% Define a custom environment for file contents
\newenvironment{file}[1][File]{ % Set the default filename to "File"
	\medskip
	\newcommand{\mdfilename}{#1}
	\begin{mdframed}[style=file]
}{
	\end{mdframed}
	\medskip
}

%----------------------------------------------------------------------------------------
%	NUMBERED QUESTIONS ENVIRONMENT
%----------------------------------------------------------------------------------------

% Usage:
% \begin{question}[optional title]
%	Question contents
% \end{question}

\mdfdefinestyle{question}{
	innertopmargin=1.2\baselineskip,
	innerbottommargin=0.8\baselineskip,
	roundcorner=5pt,
	nobreak,
	singleextra={%
		\draw(P-|O)node[xshift=1em,anchor=west,fill=white,draw,rounded corners=5pt]{%
		Question \theQuestion\questionTitle};
	},
}

\newcounter{Question} % Stores the current question number that gets iterated with each new question

% Define a custom environment for numbered questions
\newenvironment{question}[1][\unskip]{
	\bigskip
	\stepcounter{Question}
	\newcommand{\questionTitle}{~#1}
	\begin{mdframed}[style=question]
}{
	\end{mdframed}
	\medskip
}

%----------------------------------------------------------------------------------------
%	WARNING TEXT ENVIRONMENT
%----------------------------------------------------------------------------------------

% Usage:
% \begin{warn}[optional title, defaults to "Warning:"]
%	Contents
% \end{warn}

\mdfdefinestyle{warning}{
	topline=false, bottomline=false,
	leftline=false, rightline=false,
	nobreak,
	singleextra={%
		\draw(P-|O)++(-0.5em,0)node(tmp1){};
		\draw(P-|O)++(0.5em,0)node(tmp2){};
		\fill[black,rotate around={45:(P-|O)}](tmp1)rectangle(tmp2);
		\node at(P-|O){\color{white}\scriptsize\bf !};
		\draw[very thick](P-|O)++(0,-1em)--(O);%--(O-|P);
	}
}

% Define a custom environment for warning text
\newenvironment{warn}[1][Warning:]{ % Set the default warning to "Warning:"
	\medskip
	\begin{mdframed}[style=warning]
		\noindent{\textbf{#1}}
}{
	\end{mdframed}
}

%----------------------------------------------------------------------------------------
%	INFORMATION ENVIRONMENT
%----------------------------------------------------------------------------------------

% Usage:
% \begin{info}[optional title, defaults to "Info:"]
% 	contents
% 	\end{info}

\mdfdefinestyle{info}{%
	topline=false, bottomline=false,
	leftline=false, rightline=false,
	nobreak,
	singleextra={%
		\fill[black](P-|O)circle[radius=0.4em];
		\node at(P-|O){\color{white}\scriptsize\bf i};
		\draw[very thick](P-|O)++(0,-0.8em)--(O);%--(O-|P);
	}
}

% Define a custom environment for information
\newenvironment{info}[1][Info:]{ % Set the default title to "Info:"
	\medskip
	\begin{mdframed}[style=info]
		\noindent{\textbf{#1}}
}{
	\end{mdframed}
}
 % Include the file specifying the document structure and custom commands

%----------------------------------------------------------------------------------------
%	ASSIGNMENT INFORMATION
%----------------------------------------------------------------------------------------

\title{LIS4031: Project \#1} % Title of the assignment

\author{Jose Carlos Archundia Adriano - Carlos Andres Reyes Evangelista\\ \texttt{jose.archundiaao@udlap.mx - carlos.reyesea@udlap.mx}} % Author name and email address

\date{Universidad de las Am\'ericas  Puebla --- \today} % University, school and/or department name(s) and a date

%----------------------------------------------------------------------------------------

\begin{document}

\maketitle % Print the title

%----------------------------------------------------------------------------------------
%	INTRODUCTION
%----------------------------------------------------------------------------------------

\section*{Introduction} % Unnumbered section
This report aims to illustrate the implementation of the Boolean Retrieval Model, as well as the features that a retrieval model like this one might have, including its advantages, disadvantages and its inherent relevance.

The program that will be covered in this report implements a boolean retrieval model using a posting list, and was developped using \textit{Lua}, a lightweight but powerful scripting language with a built-in support for pattern matching. This feature is not as powerful as a regular expression matcher, but effective enough to perform most of the operations required to accomplish the goal natively.


\section{Boolean Retrieval Model}

\subsection{Theoretical viewpoint}
This model is one of the first retrieval models created and it is currently still being used. According to Bruce Croft et. al. (2009), this contemporary relevance is mainly due to its ease of implementation and its efficiency when it comes to the elimination of non-relevant documents during the scoring process. Nevertheless, there are several downsides to take into account when considering using it. "The effectiveness depends entirely on the user. Because of the lack of a sophisticated algorithm, simple queries will not work well", says Croft when referring to its major drawback.

The behavior of the boolean retrieval model is straightforward: the queries may contain terms and boolean operators (which may be associated) and the set of retrieved documents consists \textit{only} of documents whose content fulfills these boolean requirements. This model works under the assumption that all retrieved documents are equivalent in relevance as long as they contain the terms indicated in the query. The latter statement is the reason why this retrieval model is not considered a ranking algorithm.

\subsection{Algorithmic issues}
One of the problems with the boolean retrieval model is the fact that it uses an incidence matrix. This means that the frequency of the terms are not taken into account. In the worst case, the algorithm might be returning documents that only have one incidence of a term instead of a document with many more.
To overcome this problem, a posting list, with a frequency for each term and a list of documents in which the term appears, is used. Another problem is the fact that the program needs to parse user input for the queries to be searched. This means that user input must be validated before the query is processed.
On top of this, in order to process the query, the order in which to execute the operators of the query has to be descided by the program. This step requires extra processing of the user query to accomplish.

\section{Implementation}

\subsection*{Document parsing}
In order to generate the posting list, we first needed to extract the relevant information from the document list. To do this, we cut the document list into individual document strings by matching from one ".I" tag to another. After identifying individual documents, we then created an object in memory for each one, with fields for each one of the tags \textit{I,T,A,B,W}. By giving the data a defined structure in memory, we were then able to easily access the desired information of any one of the documents.

\subsection*{Posting list creation}
To create the posting list, we took advantage of the fact that in \textit{Lua} you can create data structures that contain key and value pairs. Each token that we found in the documents became a key in this data structure, and the value became a linked list of all the documents that contained that token. This means that we are able to rapidly able to create the posting list, and also able to rapidly access the document list for any given token.

\subsection*{Query validation and parsing}
%The program provides the user with certain freedom for how they input their query, nonetheless, that freedom needs to be supervised in order to ensure that the input query is written adequately. However, we do not want for all the non-valid queries to be rejected if the mistake was minimal.
One of the problems that the program faces when parseing the query is the validation of the user's input.
%Therefore, the algorithm that checks the validity of the query tries to \textit{correct it} by ignoring misplaced tokens and only uses the ones that may be used to form a valid query.
We want to inform the user of an error in the query when the program is unable to understandt it. At the same time however, the program tries to \textit{correct} the query by ignoring misplaced tokens and only uses the ones that may be used to form a valid query.

%     \begin{file}[Query.lua]
%         \begin{lstlisting}[language=Python]
% infixTokens = {}

% firstWordStart, firstWordEnd =
% query:find "^%(*%s*!?%(*%w+%)*"

% if firstWordStart then
%   queryHead = query:sub(firstWordStart,sfirstWordEnd)
%   queryTail = query:sub(firstWordEnd + 1, #query)

%   for _, token in pairs(getTokens(queryHead)) do
%     table.insert(infixTokens, token)
%   end

%   for pair in
%   queryTail:gmatch "[&|]%s*%(*%s*!*%s*%(*%w+%)*" do
%     for _, token in pairs(getTokens(pair)) do
%       table.insert(infixTokens, token)
%     end
%   end
% end
%         \end{lstlisting}
%     \end{file}

This algorithm basically splits the query into a head and tail. The head contains the first term of the query along with its parentheses and \textbf{not} operator, if they exist. The tail is the rest of the query, and is processed in \textit{pairs}. These \textit{pairs} only match parts of the query that contain an \textbf{and} or \textbf{or} operator followed valid combinations of spaces, parentheses, \textbf{not} operators, and an alphanumeric term. This implies that only the \textbf{valid} parts of the query are considered for future processing.

Consecuently, if the user enters a query with a incorrect syntax but valid data, the system will only consider the part that it can process. For instance, the query $standard \&\| acess$ will be interpreted as $standard \| access$ .

\subsection*{Query execution}
Evaluating an expression with operators that have a specific precedence is not a simple task, since the program is required to evaluate certain parts of the expression first, independently of where are they found. To address this issue, the query was transformed from an infix-form (more human-readable) to a postfix-form (more machine-readable).

This procedure is not complicated; using the \textbf{valid} expression generated above, the steps are as follows:
A stack is declared to temporarily store the operators and a list that stores the pre-constructed postfix form. Each infix token is then processed one by one. If it is a term, it is immediately pushed into the postfix list. If it is an opening parenthesis, it is immediately pushed into the operators stack. If it is any other operator, then its precedence is compared to the one on the top of the stack. Operators with a greater precedence are pushed, otherwise, all the operators precedence greater than or equal to [?] are popped from the stack and pushed into the list. Finally, if the processing token is a closing parenthesis, then all the operators in the stack are popped until the next opening parenthesis is found. If an opening parenthesis is found when all the infix tokens are processed, or if a closing parenthesis is found before an opening one, then an exception is returned.

Once the postfix form is obtained, it is processed by pushing all the terms into a new stack. Every time an operator is found, the required amount of parameters are popped from the stack and used to perform that operation, and will push the result into the stack. The remaining term in the stack is the final result of the expression.

\section{Conclusions}
Throughout the development of a program that implements the Boolean Retrieval model using a posting list, we were able to come to understand the nuances of using this approach.
%We saw that this method requires the user to enter sophisticated queries in order to get pertinent results, and were also able to experience the strengths and weaknesses that this model offers.
We were able to experience firsthand the strengths and weaknesses of this model.
Our implementation of the Boolean Retrieval Model allowed us to deepen our understanding of it, and helped to solidify our knowledge on the topic.

\section{References}
Croft, W., Metzler, D. and Strohman, T. (2009). \textit{Information retrieval in practice.} Upper Saddle River, N.J.: Pearson Education.

%----------------------------------------------------------------------------------------

\end{document}
